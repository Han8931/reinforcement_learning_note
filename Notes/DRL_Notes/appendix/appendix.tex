\renewcommand{\thesection}{\Alph{section}.\arabic{section}}
\setcounter{section}{0}

\begin{appendices}
\chapter{Appendix}

\section{Bellman Equation}

\textit{Bellman equation} can be derived as follows:

\begin{align*}
	v_\pi(s) &= \mathbb{E}_\pi[G_t|S_t=s]\\
	& = \mathbb{E}_\pi\Bigg[\sum_{k=0}^{\infty}\gamma^k R_{t+k+1}\Big|S_t=s\Bigg]\\
	& = \mathbb{E}_\pi[R_{t+1} + \gamma R_{t+2} + \gamma^2 R_{t+3} + ...|S_t=s]\\
	& = \mathbb{E}_\pi[R_{t+1} + \gamma (R_{t+2} + \gamma R_{t+3} + ...)|S_t=s]\\
	& = \mathbb{E}_\pi[R_{t+1} + \gamma G_{t+1}|S_t=s]\\
	& = \mathbb{E}_\pi[R_{t+1}|S_t=s] + \gamma \mathbb{E}_\pi[ G_{t+1}|S_t=s]\\
	& = \mathbb{E}_\pi[R_{t+1}|S_t=s] + \gamma \mathbb{E}_\pi\Big[\mathbb{E}_\pi[ G_{t+1}|S_{t+1}=s']\Big|S_t = s_t\Big] \\
	& = \mathbb{E}_\pi[R_{t+1}|S_t=s] + \gamma \mathbb{E}_\pi\Big[v(s_{t+1})\Big|S_t = s_t\Big]\\
	& = \mathbb{E}_\pi[R_{t+1} + \gamma v(s_{t+1})|S_t=s]\\
	& = \sum_{a}\pi(a|s)\sum_{s',r}p(s',r|s,a)[r + \gamma v_\pi(s')]
\end{align*}
The expectation here describes what we expect the return to be if we continue from state s following policy $\pi$. The expectation can be written explicitly by summing over all possible actions and all possible returned states. The next two equations can help us make the next step.

\href{https://stats.stackexchange.com/questions/243384/deriving-bellmans-equation-in-reinforcement-learning}{Reference}

\end{appendices}
